\documentclass[15pt]{ctexart}
\usepackage{xeCJK}
	\setCJKmainfont{SimSun}         % 缺省中文字体为宋体
	\setmainfont{Times New Roman}   % 缺省英文字体 Times New Roman
\usepackage{geometry}
	% \geometry{left=2.5cm,right=2.5cm,top=2.5cm,bottom=3.5cm}
	\geometry{bottom=3.5cm}
\usepackage{float}
\usepackage{listings}
\usepackage{graphicx}
\usepackage{appendix}
\usepackage[colorlinks]{hyperref}
\usepackage{amsmath}
\usepackage{indentfirst}
\usepackage{enumerate}
\usepackage{fancyhdr}
\usepackage{textcomp}
\usepackage{appendix} 
\usepackage{multirow}
\usepackage{geometry}
\geometry{verbose,letterpaper}
\usepackage{media9}
\pagestyle{fancy}
\lhead{虚拟路由}
\rhead{\thepage}
\cfoot{}
\lstset{frameshape={RYRYNYYYY}{yny}{yny}{RYRYNYYYY}, backgroundcolor=\color[RGB]{245,245,244}}

\begin{document}
\begin{titlepage}
    \centering
    \includegraphics[scale=0.9]{imgs/SYSULogo.png}\par\vspace{1cm}
    \vspace{1cm}
    {\scshape\huge 虚拟路由 \\ 项目报告 \\ \centering \scshape \Huge 模拟网络层路由 \par}
    \vspace{1.5cm}
    {\Large\bfseries \flushleft 学院:数据科学与计算机学院 \\ 专业:计算机科学与技术 \\ 年级:2016级 \\组长(学号):王锡淮(16337236)\\组员(学号):杨陈泽(16337271)\\
    组员(学号):肖遥(16337258)\par}
    % \vspace{2cm}

% Bottom of the page
    % {\large \today\par}
\end{titlepage}
\tableofcontents
\newpage

\begin{table}[H]
	\centering
	\begin{tabular}{|c|}
		\hline
		Command \; \vline \; Version \; \vline \; Routing domain \; \\
		\hline
		Source address \\
		\hline
		Address family \; \vline \; Route tag \; \\
		\hline
		address \\
		\hline
		Next hop \\
		\hline
		metric \\
		\hline
		repeat of last 17 bytes \\
		\hline
		$\cdots$ \\
		\hline 
	\end{tabular}	
\end{table}
报文项解释:
\begin{enumerate}
	\item Command(1 Byte):指明这条报文的类型。
	\begin{enumerate}[]
		\item 0:普通报文。
		\item 1:请求报文。
		\item 2:响应(包含路由表)报文。
		% \item 3:广播request。
		% \item 4:广播response。
	\end{enumerate}
	\item Version(1 Byte):RIP协议的版本。
	\item Routing domain(2 Bytes):指明这是RIP中的哪一步。
	\item Source address(6 Bytes):指明发送方的源地址和监听端口。
	\item Address family(2 Bytes):使用什么作为地址,使用IP该项为2。
	\item Route tag(2 Bytes):本项目中不用。
	\item address(6 Bytes):ip和端口号。
	\item Next hop(6 Bytes):下一跳路由的ip和端口。
	\item matric(1 Bytes):代价度量(跳数)。
\end{enumerate}
\begin{table}[H]
	\centering
	\begin{tabular}{|c|}
		\hline
		Command \; \vline \; Version \; \vline \; Routing domain \; \\
		\hline
		Source address \\
		\hline
		Address family \; \vline \; Route tag \; \\
		\hline
		address \\
		\hline
		Next hop \\
		\hline
		metric \\
		\hline
	\end{tabular}
\end{table}
报文项的内容和response表项一致。
\begin{table}[H]
	\centering
	\begin{tabular}{|c|}
		\hline
		Command \\
		\hline
		source address \\
		\hline
		destination address \\
		\hline
		payload \\
		\hline
	\end{tabular}		
\end{table}
其中的源地址和目的地址包含ip和监听端口号。
% \par 需要特别指明的是当用于广播时,目的ip写的是224.0.0.9。
\end{document}